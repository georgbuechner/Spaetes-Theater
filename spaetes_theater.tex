\documentclass[12pt, a4paper, openany]{report}
\usepackage[left=3cm, top=3cm, bottom=3cm, right=4cm]{geometry}

\usepackage{mystyle}
\usepackage{csquotes}

\pagestyle{fancy}
\fancyhf{}
\lhead{Partykomitee}
\chead{Spätes Theater}
\rhead{\thepage}

\title{
    {\textbf{Spätes Theater}}\\
    {\large{Eine Bildungsliverollenspielparty}}\\
}
\author{Tristan, Ronja, Anna, Jan}
\date{\today}

\begin{document}


\maketitle
\frontmatter
\tableofcontents
\mainmatter

\chapter{Einleitung}

\chapter{Spielsetting}

\section{Zeitalter und Gefolgenheiten}

\section{Einladung für die Teilis/ grund der Zusammenkunft}
Liebe Freunde und Vertraute,\\
wir leben in harten Zeiten. 
Um dies aufuklären will ich dich klügsten Köpfe unserer Zeit zusammenrufen. 
Wir werden uns versammeln, um aufzudecken, wer die Protokolle der Weisen von Zion veröffentlich hat. 

\chapter{Plot}

\section{Ronja}
Alle kennen den Veranstalter. 
Er hat großen politischen Einfluss. 
Am Anfang wird eine kurze Eröffnungsrede gehalten in der vier Leute vorgestellt werden, die als langjährige Vertraute des Veranstalters vorgestellt werden (vermutlich seine Diener).
Sie haben ein Projekt im Kopf und das wird im Laufe des Abends herumgetragen. 
Sie werden gleichzeitig durch den Abend die Gäste bewirtschaften und als Diener agieren.
Zwei davon sind gut und zwei insgeheim böse. 
Der Veranstalter weiß es nicht. 
Die Bösen wissen, wer dessen Verbündeter ist.
Die bösen haben somit einen Teamspieler. 
Im Verlaufe der Veranstaltung reden die vier Diener mit allen Gästen der Veranstaltung und versuchen sie für ihre Meinung zu gewinnen. 
Das soll son bisschen Werwolf-mäßig sein. 
Die Gäste müssen aus vorgeschriebenen Zetteln unterschreiben, auf denen ein konkreter Hinweis steht.
Z.B.: \qq{Wir sind gegen die Konzentration des Kapitals in den Händen von Hackennasen} (alternativ: \qq{von Puppenspielern})
Die Protokolle von Zion werden vorher kommuniziert. 
Beispielsweise dass wir dagegen sind, dass sich in bestimmten Händen kein Geld gelangen darf.\\

\section{Tristan}
Gehobene Veranstaltung von Wissenschaftlern, Künstlern, Politikern und Individuen.\\
Die Protokolle der Wisen von Zion sind entstanden wodurch die Juden als Böse Macht dargestellt werden. 
Die Gruppen der Veranstaltung arbeiten entweder für die Unterzeichnung dieser Protokolle oder dagegen.
Sollten sie unterschrieben werden werden böse und zerstörerische Interessen zur Wahrheit.
Wenn nicht, dann nimmt die Geschichte ihren Lauf und wir erreichen die uns bekannte Gegenwart.
Es geht also darum, die Vergangenheit so zu beeinflussen, dass alle zukünftigen Ereignisse geschehen können.
Beispiele: Relativitätstheorie wird entdeckt, Radioaktivität wird entdeckt, unglaubliche Kunstwerke entstehen, die unsere heutige Welt stark beeinflusst.\\
Sollten die Protokolle unterschrieben werden, würde die Jüdische Gemeinde als Böses Volk dargestellt.\\
Sollten sie nicht unterschrieben werden, würden sie inn Zukunft nicht gehasst werden und sie könnten Ihren Beitrag zur Weltgeschichte bringen.\\\\\\
Über den Verlauf der Veranstaltung laufen 4 Kellner herum, die Unterschriften für ein Dokument sammeln. 2 dieser Kellner sind neutral und 2 haben ein böses Motiv.
Alle Gäste

\qq{Wir sind gegen die Konzentration des Kapitals in den Händen von Hakennasen}

\chapter{Events}
\section{Präsentation}
\section{Tanzen}
Wir brauchen noch geile und passende Musik um zu tanzen:
Jazz, Swing z.B.

\section{Glücksspiel}
Eventuell als Verantworliche/r : Michelle und Don

\chapter{Gruppen}
\section{Die Liebes Affäre}
\subsubsection{Backgroundstory}
\subsubsection{Hinweise für uns}
Vielleicht will diese Gruppe auch Menschen auseinander bringen.\\\\
Beispiel: Simone de Barvoix (Frau von Sarte) mit jemandem verkuppeln\\
Beispiel: Love Triangle zwischen Nietzsche, Solme unnd Ree
\subsubsection{Ziel}
Einige Charaktere wollen Character XY mit Charakter XZ verkuppeln.
Allerdings wollen ein bis drei Menschen dieser Gruppe Charakter XY mit Charakter XW verkuppeln.

\section{Seduction}
\subsubsection {Backkgroundstory}
Die Künstler*innen haben zum einen reguläre Künstler*innen-Skepzis gegen die Wissenschaftler, außerdem haben aber (einige) von ihnen Angst vor der fortschreitenden Technik (Atombombenrezept der Physiker*innen).

\subsubsection {Ziel}
Die Künstler und Schriftstelle wollen die Wissenschaftler mit Alkohol zum Rausch verführen.
\subsubsection {Hinweise für uns }

\section{Politische Intrigen}
\subsubsection{Backkgroundstory}
\subsubsection{Ziel}
Diese Gruppe will ihr Dokument gegen den aufkeimenden Nationalsozialisten besprechen und \emph{alle} Menschen auf der Veranstaltung dazu bringen dieses Dokument zu unterschreiben.

\subsubsection{Hinweisee für uns}
\emph{Für diese Gruppe gibt es ein IT-Vortreffen, welches vor dem Spiel stattfindet.
Das Treffen soll ohne sozial Media o.ä. kommuniziert werden.}

\section{Weltherrschaft}
\subsubsection{Backgroundstroy}
Möglicherweise ein grausiger Plan die Weltherrschaft zu ubernehmen. 
Ein haufen militaerischer und dogmatischer unantastbarer Egozentriker (nein nicht Hitler) die mithilfe von machiavelistischen Mitteln die Weltherrschaft an sich reissen moechten. 

\subsubsection{Ziel}
Ihr Ziel ist es die geheimen Dokumente der Physiker zu erlangen, um die Baupläne für die Atombombe zu erhalten.

\subsubsection{Hinweise für uns}
Evtl. könnte diese Gruppe, um ihr Ziel zu erreichen, sich mit der Gruppe \qq{Seduction} zusammentun, um die abzufüllen.

\chapter{Charactere}
Hier können wir alle Charactere Versammeln.

\section{Figuren der Verannstaltung}
\begin{itemize}
    \item Butler
    \item Diener
    \item Veranstalter
\end{itemize}

\section{Mögliche Gast-Charaktere}
Vorschlaege von Tristan
\subsubsection{Schriftsteller}
\begin{itemize}
    \item Ernest Hemingway (geboren 1899)
    \item C. S. Lewis (geboren 1898)
    \item Jean-Paul Sartre
    \item Simone de Barvoiox
    \item Friedrich Nietzsche
    \item Lou Andreas Salomè 
    \item Paul Rèe 
\end{itemize}

\subsubsection{Wissennschaftler}
\begin{itemize}
	\item Max Planck
    \item Heisenberg
    \item Schrödiger
    \item Albert Einstein
	\item Leibnitz (Optimist)
\end{itemize}

\subsubsection{Künstler}
\begin{itemize}
	\item Alfred Hitchcock
	\item Frieda Kahlow
    \item Käthe Kollwitz
\end{itemize}

\subsubsection{Andere Charaktere}
\begin{itemize}
	\item Enzo Ferrari (geboren 1898)
	\item Rudolf und Adolf Dassler (geboren 1998 und 1900)
	\item Siegmunnd Freud
\end{itemize}

\section{Characterbeschreibungen}
Das sind die Beschreibungen, die die Teinehmer*innen auch erhalten.

\chapter{OT-Mechanismen}

\chapter{Regeln}
\section{Schießt euch nicht ab}
Es wird IT-Alkohol zur Verfügung gestellt, bitte konsumiert während dem Spiel nichts zusätzliches, was darüber hinausgeht.
OT-Drogen: macht was ihr wollt, aber achtet darauf, dass ihr euch nicht abschießt, hebt euch eure Emmas und Lucys auf für nach dem Spiel.

\chapter{Orga}
\section {bridges}
\begin{itemize}
    \item wir können notfalls einen der physiker spielen, zum Beispiel, dass wenn wir merken, dass sie es mit dem Skript nicht auf die reihe kriegen, dass wir der Machtgruppe helfen und einen kleinen hint geben. Es gibt fünf Teile des Skripts.
\end{itemize}
\section{OT-Raum}
Wir brauchen einen OT Raum, der bisher folgende Funktionen erfüllen soll:
\begin{itemize}
    \item Teilnehmer können ihre Sachen dort ablegen.
    \item Wir können uns besprechen, um sponan Entscheidungen zu treffen.
    \item Teilnehmer*innen können den OT-Raum als Rückzugsort benutzen. \todo{wer ist immer im OT Raum?}
\end{itemize}

\chapter{Random Ideen}
\begin{itemize}
    \item Livemusik von Tim, Vasco und Antonio.
    \item Zu Beginn der veanstaltung müssen alle eine Unterschrift abgeben. 
Damit können im Nachhinein evtl. Unterschriftfälscher entlarvt werden, bzw. Menschen können Unterschriften fälschen.
    \item In der Characterbeschreibung verstecktes Passwort, dass bei Telefonbriefing abgefragt wird.
\end{itemize}

\chapter{Briefing}
\begin{itemize}
    \item Wir wollen jeder Person mitgeben, dass sie essentieller Bestandteil der Veranstaltung ist.
    \item Dazu aufrufen mal nach deinem Charakter zu googlen.
    \item Einige Textvorschläge schicken mit denen sich auseinandergesetzt werden könnte, um tiefer in den Charakter einsteigen zu können.
    \item Telefonbriefings für jeden in denen das (versteckte) Passwort abgefragt wird.
\end{itemize}

\chapter{Genres}
Folgende Generes kommen in dem Konzept bisher vor, welche wollen wir noch einbauen? 
Wie wollen wir den Flair von Theaterstücken/ Filmen aufnehmen?
\begin{itemize}
    \item[Krimi] Hauptthema: war hat die Protokolle der Weisen von Zion veröffentlicht?
    \item[Romanze] Gruppe 1: Liebes Aphäre
\end{itemize}

\chapter{Lerninhalte}
\section{Links zu (strukturellem) Antisemitismus}
\begin{itemize}
    \item \href{https://www.bpb.de/politik/extremismus/antisemitismus/285539/antisemitismus-im-deutschsprachigen-rap-und-pop}{BPB - Antisemitismus im deutschsprachigen Rap}
    \item \href{https://www.bpb.de/politik/extremismus/antisemitismus/37974/antisemitismus-heute}{BDP - Antisemitismus heute}
\end{itemize}

\section{Sozial Media}
\section{Warum Briefe}
\section{Antisemitismus und die Protokolle der Weisen von Zion}

\chapter{Reflexion}
\section{Kurzes Debriefing}
Körper ausschütteln/ Rolle abwerfen.

\section{Stumme Diskussion}
Nach dem Spiel werden Plakate zu verschiedenen Themen/ Fragen aufgehängt, à la \qq{Was hat euch gestört?}, \qq{Was fandet ihr mega?}... Auf denen \qq{stumm} diskutiert werden kann.

\section{Reflexionstreffen}
Eine Woche später Reflexionstreffen mit interessierten Menschen.

\chapter{Material}
\begin{itemize}
    \item Unterschriftenzettel erstellen.
\end{itemize}

\printbibliography
\listoftodos
\end{document}

    

