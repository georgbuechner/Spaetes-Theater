\documentclass[12pt, a4paper, openany]{report}
\usepackage[left=3cm, top=3cm, bottom=3cm, right=4cm]{geometry}

\usepackage{mystyle}
\usepackage{csquotes}

\pagestyle{fancy}
\fancyhf{}
\lhead{Partykomitee}
\chead{Spätes Theater}
\rhead{\thepage}

\title{
    {Spätes Theater}\\
    {\large{Eine Bildungsliverollenspielparty}}\\
}
\author{Tristan, Ronja, Anna, Jan}
\date{\today}

\begin{document}


\maketitle
\frontmatter
\tableofcontents
\mainmatter

\chapter{Einleitung}

w

\chapter{Spielsetting}

\section{Zeitalter und Gefolgenheiten}

\section{Einladung für die Teilis/ grund der Zusammenkunft}
Liebe Freunde und Vertraute,\\
wir leben in harten Zeiten. 
Um dies aufuklären will ich dich klügsten Köpfe unserer Zeit zusammenrufen. 
Wir werden uns versammeln, um aufzudecken, wer die Protokolle der Weisen von Zion veröffentlich hat. 

\chapter{Plot}
\chapter{Events}
\section{Präsentation}
\section{Glücksspiel}
Eventuell als Verantworliche/r : Michelle und Don

\chapter{Gruppen}
\section{Die Liebes Afäre}
Einige Charaktere wollen Character XY mit Charakter XZ verkuppeln.
Allerdings wollen ein bis drei Menschen dieser Gruppe Charakter XY mit Charakter XW verkuppeln.
Vielleicht will diese Gruppe auch Menschen auseinander bringen.\\\\
Beispiel: Simone de Barvoix (Frau von Sarte) mit jemandem verkuppeln\\
Beispiel: Love Triangle zwischen Nietzsche, Solme unnd Ree

\section{Seduction}
Die Künstler wollen die Wissenschaftler mit Alkohol zum Rausch verführen. 

\section{Politische Intrigen}
Diese Gruppe will ihr Dokument gegen den aufkeimenden Nationalsozialisten besprechen und \emph{alle} Menschen auf der Veranstaltung dazu bringen dieses Dokument zu unterschreiben.\\

\emph{Für diese Gruppe gibt es ein IT-Vortreffen, welches vor dem Spiel stattfindet.
Das Treffen soll ohne sozial Media o.ä. kommuniziert werden.}

\section{Weltherrschaft}

M"oglicherweise ein grausiger Plan die Weltherrschaft zu ubernehmen. 
Ein haufen militaerischer und dogmatischer unantastbarer Egozentriker (nein nicht Hitler) die mithilfe von machiavelistischen Mitteln die Weltherrschaft an sich reissen moechten. 
Das koennten die geheimen Gegner der Party sein und das waere unser "Mord" im Orient Express. 
So kommt auch der Antisemitismus zum Vorschein und so findet unser Oberthema (Obermythe) die Protokolle der Weisen von Zion einen Sinn.\\
IT koennen sie in Gespraechen Schaden anrichten und OT kann die Part sabotiert werden. 


\chapter{Charactere}
Hier können wir alle Charactere Versammeln.

\section{Figuren der Verannstaltung}
\begin{itemize}
    \item Butler
    \item Diener
    \item Veranstalter
\end{itemize}

\section{Mögliche Gast-Charaktere}
Vorschlaege von Tristan
\subsection{Schriftsteller}
\begin{itemize}
    \item Ernest Hemingway (geboren 1899)
    \item C. S. Lewis (geboren 1898)
    \item Jean-Paul Sartre
    \item Simone de Barvoiox
    \item Friedrich NIetzsche
\end{itemize}

\subsection{Wissennschaftler}
\begin{itemize}
	\item Max Planck
	\item Die Physiker (Dürrenmatt)
	\item Leibnnitz (Optimist)
\end{itemize}

\subsection{K"unstler}
\begin{itemize}
	\item Alfred Hitchcock
\end{itemize}

\subsection{Andere Charaktere}
\begin{itemize}
	\item Enzo Ferrari (geboren 1898)
	\item Rudolf und Adolf Dassler (geboren 1998 und 1900)
	\item Siegmunnd Freud
	\item Albert Einstein
\end{itemize}
\section{Characterbeschreibungen}
Das sind die Beschreibungen, die die Teinehmer*innen auch erhalten.

\subsection{Nietzsche}
\subsubsection{Hintergrundgeschichte}
\subsubsection{Beziehungen}
\subsubsection{Psychologie}
\subsubsection{Deine Ziele}


\chapter{OT-Mechanismen}

\chapter{Regeln}
\section{Schießt euch nicht ab}
Es wird IT-Alkohol zur Verfügung gestellt, bitte konsumiert während dem Spiel nichts zusätzliches, was darüber hinausgeht.
OT-Drogen: macht was ihr wollt, aber achtet darauf, dass ihr euch nicht abschießt, hebt euch eure Emmas und Lucys auf für nach dem Spiel.

\chapter{Orga}

\section{OT-Raum}
Wir brauchen einen OT Raum, der bisher folgende Funktionen erfüllen soll:
\begin{itemize}
    \item Teilnehmer können ihre Sachen dort ablegen.
    \item Wir können uns besprechen, um sponan Entscheidungen zu treffen.
    \item Teilnehmer*innen können den OT-Raum als Rückzugsort benutzen. \todo{wer ist immer im OT Raum?}
\end{itemize}

\chapter{Random Ideen}
\begin{itemize}
    \item 
\end{itemize}
\chapter{Genres}
Folgende Generes kommen in dem Konzept bisher vor, welche wollen wir noch einbauen? 
Wie wollen wir den Flair von Theaterstücken/ Filmen aufnehmen?
\begin{itemize}
    \item[Krimi] Hauptthema: war hat die Protokolle der Weisen von Zion veröffentlicht?
    \item[Romanze] Gruppe 1: Liebes Aphäre
\end{itemize}

\chapter{Lerninhalte}
\section{Sozial Media}
\section{Warum Briefe}
\section{Antisemitismus und die Protokolle der Weisen von Zion}

\printbibliography
\listoftodos
\end{document}

    

